\documentclass[10pt,a4paper,landscape]{article}
%
% Spaltenbreite soll 8cm sein (passt genau in AA Schachteln)
% A4 = 297x210, benutzen landscape
% 29,7/8 = 3,7125 -> 3 Spalten à 8cm + 5,7cm Rand..
% Sperator: 0,5cm
% margin-hori: 3,1
% Spaltenbreite 8cm -> 7,5 text 0,5 padding
% => 3*7,5 + 2*0.5 + 2*3,1 = 29,7 :)
%

\usepackage[landscape,
			hmargin={3.1cm},
			vmargin={0.8cm, 1.2cm}]{geometry}
\usepackage[utf8]{inputenc}
\usepackage[german]{babel}
\usepackage[T1]{fontenc}
\usepackage{multicol}

\setlength{\columnsep}{0.5cm}
\setlength{\columnseprule}{0.4pt}

\begin{document}
\begin{multicols*}{3}

\section*{Turn Sequence}
\begin{enumerate}
\item Purchase and Repair Units/Facilities\\
ship repairs are free, facilities cost 1 IPC per damage point 
\item Combat Move
\begin{enumerate}
\item Declare war?
\item Announce combat movements
\begin{enumerate}
\item Amphibious assaults
\item Strategic and tactical bombings
\item Normal combat moves
\end{enumerate}
\item Defender scrambling?
\end{enumerate}
\item Conduct Combat
\begin{enumerate}
\item Defender Kamikaze?
\item Strategic and tactical bombing raids
\item Amphibious Assault
\item General combat
\end{enumerate}
\item Noncombat Move\\
Moving troops and landing air units that engaged in battle (allowed even on carriers being build this turn phase 5).
\item Mobilize New Units\\
Place bought units from phase 1 on the map. If you bought more units than you can place (eg. because of facility damage), you have to give units back. Price is refunded. Before placement check industry damage.
\item Collect Income
\begin{enumerate}
\item Convoy disruptions
\item Check National Objectives
\item Calculate IPC for next round
\end{enumerate}
\end{enumerate}
\columnbreak

% title 
\begin{center}
{\Large Short reference}\\
\vspace{.2cm}
{\large Axis \& Allies 1940 Pacific 2nd Edition}

\vspace{.5cm}
contribute: https://github.com/lensz/aa-short-reference
\end{center}

\section*{{\normalsize Sources}}
\begin{itemize}
\item Rulebook 2013 version
\item FAQ November 24, 2014
\end{itemize}

\noindent\textsl{Note that columns fit in the actual AA boxes..}

\section*{{\normalsize Abbreviations}}
\textbf{JP} Japan ~~ \textbf{US} USA ~~ \textbf{UK} United Kingdom ~~ \textbf{AZ} ANZAC ~~ \textbf{CH} China ~~ \textbf{UKAZ} UK and AZ



\noindent\rule{7.5cm}{0.4pt}

\subsection*{General Combat}
\begin{enumerate}
\setcounter{enumi}{-1}
\item AAA's fire (only round 1) up to three times
\item Submarine surprise strike or submerge
\item Attacker fires
\item Defender fires
\item Push attack (goto 1) or retreat?
\end{enumerate}

\section*{Political Situation}
\begin{description}
\item[Axis Victory] JP controlling 6 victory cities (including Tokyo) for a complete round of play (So in beginning of JP turn). Liberation of Tokyo in this turn is also allowed.\textsuperscript{p.24}
\item[Allied Victory] Allies control one of their capitals plus Tokyo for complete round. Liberation  of a capital is also allowed. \textsuperscript{p.24}
\end{description}

\subsection*{Politics}
At the beginning, there is only war between JP-CH. Allied troops in/through CH are an act of war against JP. CH units can only move on CH emblems plus Kwangtung and Burma - even the fighter. No power on either side may enter sovjet territory.

If JP is at peace with US, JP may not end move within 2 territories of Alaska and West US. On the other hand, US may not end move in sea zone adjacent to JP-controlled territories. 

An unprovoked attack from JP to \{AZ, UK, Dutch\} results in war with whole Allies. And US gets 30 IPC immediately. If this does not happen, US can declare war on JP if both are still at peace at the beginning of the \glqq Collect Income\grqq\ phase in the third US turn. So in the end, US has every round 30 IPC extra since the first turn they are at war independetly from the type of war declaration.

JP attack on France has no consequences here. UKAZ is treated as one nation for political purposes. UKAZ may declare war on JP any time (but then without US).
UKAZ may take dutch territories for free right from the beginning.

\subsection*{Movement misc.}
\begin{itemize}
\item submarines/transports don't block movement like surface warships. But you have to ignore them all at once or attack them all and stop moving
\item submarines can move in/through hostile seazones if there is no destroyer present
\item fly over neutrals is not allowed until captured (even friendly neutral)
\end{itemize}

\subsection*{Blitzing}
Move two territories through hostile lands and capture the first one on the fly. You can only blitz through empty territory. Units - even AAA, airports and industry - stop the movement. \textsuperscript{p.14}

\subsection*{Surprise Strike and Submerge of submarines}
Attacking and defending submarines may choose one of the above before at every round of the sea battle - but only if the opposing side has no destroyer! Attacking player decides first what to do. Submarines that did a surprise strike don't fire in the normal combat again.

\subsubsection*{Submerge}
Submarines are immediately removed from battle, back to the map.

\subsubsection*{Surprise Strike}
Attacking surprise strike hits on a 2 or less. Defending surprise strike hits on 1 or less. Can only hit sea units no air units. Destroyed units are removed before residual combat round. Trapos can't be hitten, unless they are the only target. Can be done at every combat round as long as there is no destroyer opposing.

\subsection*{Scramble}
Is a move of the defender at the end of the \glqq Combat Move\grqq\ of the attacker. Attacker may not change something after scrambling is announced.

Scrambled planes may defend against enemy ships even if there are no friendly ships present. Friendly powers can scramble, too. With respect to the normal three aircraft per airbase rule. \textsuperscript{p.15}

\subsection*{Kamikaze}
Special defense strike against surface warships. Cannot hit transports and submarines. JP must announce before conducting combat. JP must declare which ship is attacked with how many strikes. Every token hits on a 2 or less. Successful or not, the strike prevents offshore bombardment. \textsuperscript{p.16}

\subsection*{Amphibious Assaults}
All amphibious assaults have to be announcend before any combat. Attacker decides first which of his planes fight on sea and which on land. Later defender decides as usual if he scrambles. Trapos can only unload from a friendly seazone (no enemy surface warships). While enemy submarines and trapos may be ignored, your Trapos can only unload in sea zones with ignored submarines if there is at least one friendly warship present.
\begin{enumerate}
\item Sea Combat\\
With defending warships and/or scrambled planes. Defending submarines/trapos may be ignored. But if there is combat, all units are involved.
\item Shorebombardement\\
Only if there was NO sea combat in the offloading zone.  One ship per unloaded landunit may fire. Normal dice rolls for battleship and cruiser. Defeated landunits may fire back in the first land combat round!
\item Land combat\\
Normal land combat, but seaborne units cannot retreat
\end{enumerate}

\subsection*{Capturing/Liberating Capitals}
Liberate captured territory from your alliance gives it to the original controller if his capital is not in enemy hands. If capital is captured, you take it for yourself until the capital is free.

\textbf{Capture Capital:} Like normal territory. Plus you get all unspend IPC of the country. Original controller can't collect income from territories and can't buy new units. He does only \glqq Combat\grqq\ and \glqq Non-Combat\grqq\ until his capital is liberated.

\textbf{Liberate Capital:} territories in hands of friends go immediately back to own control.

\subsection*{Strategical and Tactical Bombing Raids}
Direct attack with tac.-bomber on air- or navalbase. Str.-bomber can attack industry, too.
Damaged facilities have to be repaired (with IPC). Defender and Attacker can both add fighters (but they don't have to). Defending fighters come from attacked territory - Attacking from everywhere. If defender uses fighters, a special airbattle follows. 
\begin{itemize}
\item one round of combat
\item all units have attack/defense of one
\end{itemize}
Fighters don't participate in the actual bombing, so no AA fire on them. If territoy has multiple targets, the bombers may be split. Each facility has its own AA.\\
Each complex rolls one dice against each bomber attacking it. Hits on 1 or less and the bomber is removed immediately. Now the surviving bombers roll one dice each. Str. bombers add 2 to each dice. Sum of those is the total damage. Place one gray chip under the facility for every damagepoint to indicate damage. Max damage is 20 for major industry, 6 for everything else.\\
No plane involved can participate in other battles in that turn. \textsuperscript{p.16}

\subsection*{Convoy Disruptions}
At the beginning of your collect-income phase, do enemy warships (all except carrier) and carrier based air units fire on your convoys. Only in convoy-provinces. Your own warships in the zone have no effect.\\
Submarines and air units have two dices, all other warships only one. Hits 4 or more are ignored. The total damage is the sum of all rolls with 3 or less. But you can only disrupt so much IPC as the neighbouring territories have!

\pagebreak

\section*{Units}
\subsection*{Industry}
\texttt{Cost: minor 12, major 30, upgrade 20}\\
For building units.\\
On capture major one get downgraded to minor.
Minor needs territory with value two or higher. Major require 3 or higher and original controlled territory. Both cannot be placed on islands. You cannot build on friendly industry (even liberated)\\
Damaged industry produces less units (one for every damage point). Maxdamage is 6 and respectively 20.

\subsection*{Air- and Navalbase}
\texttt{Cost: 15}\\
One additional movement point, large ship repair and allowes scrambling (air).\\
Carrier Aircraft does not profit from airbase on land. Carriers and battleships get repaired in friendly navalbases in the turn of the shipowner.
\\
If has three or more damage points gets inoperative. So range is reduced, no repairs are done and scrambling is prohibited.

\subsection*{Infantry}
\texttt{Cost: 3 \\ Attack: 1 (2 with Art) \\ Defense: 2 \\ Move: 1}
\\
Attack increases if paired with artillery.

\subsection*{Artillery}
\texttt{Cost: 4 \\ Attack: 2 \\ Defense: 2 \\ Move: 1}
\\
Can be paired with infantry and mechanized infantry

\subsection*{Mechanized Infantry}
\texttt{Cost: 4 \\ Attack: 1 (2 with Art) \\ Defense: 2 \\ Move: 2}
\\
Attack increases if paired with artillery.\\
Can blitz when paired with a tank. Moving must start and end in same province for tank and mech.

\subsection*{Tanks}
\texttt{Cost: 6 \\ Attack: 3 \\ Defense: 3 \\ Move: 2}
\\
Can be paired with tac. bombers and mech. infantry.\\
Can blitz. 

\subsection*{AAA (Anti Aircraft Artillery)}
\texttt{Cost: 5 \\ Attack: - \\ Defense: - \\ Move: 1 (only non-combat move)}
\\
Can not be used during attacks (except amphibious assault). Can not defend, but can be used as casualty.\\
Can defend against planes attacking the territory it stands in. Fire only once before the first actual combat round. Hits on 1. Each AAA may fire on 3 attacking airplanes, but on each airplane may only be shot once (So 5 fighters, 2 AAA $\rightarrow$ 5 rolls).

\subsection*{Fighters}
\texttt{Cost: 10 \\ Attack: 3 \\ Defense: 4 \\ Move: 4}
\\
Can be used on a carrier.\\
Can escort and intercept bombing raids (see \glqq Strategical and Tactical Bombing Raids\grqq)

\subsection*{Tactical Bombers}
\texttt{Cost: 11 \\ Attack: 3 (4 with tank or fighter) \\ Defense: 3 \\ Move: 4}
\\
Can be used on a carrier.\\
Can be paired with fighters or tanks to increase attack value to 4.\\
Can conduct bombing raids (see \glqq Strategical and Tactical Bombing Raids\grqq).

\subsection*{Strategic Bombers}
\texttt{Cost: 12 \\ Attack: 4 \\ Defense: 1 \\ Move: 6}
\\
Can not be used on a carrier.\\
Can conduct bombing raids (see \glqq Strategical and Tactical Bombing Raids\grqq).


\subsection*{Battleships}
\texttt{Cost: 20 \\ Attack: 4 \\ Defense: 4 \\ Move: 2}
\\
Has two lifes.\\
Can conduct offshore bombardment.

\subsection*{Carrier}
\texttt{Cost: 16 \\ Attack: 0 \\ Defense: 2 \\ Move: 2}
\\
Has two lifes.\\
Guest planes are treated as cargo and move with the carrier but don't defend it. Own planes move independently from the carrier. So, carriers are airports in the sea zone they end their move.\\
Damaged carriers can't conduct air operations. Planes can't land on a damaged one. Friendly (cargo) planes are trapped on it and cannot move.

\subsection*{Cruiser}
\texttt{Cost: 12 \\ Attack: 3 \\ Defense: 3 \\ Move: 2}
\\
Can conduct offshore bombardment.

\subsection*{Destroyer}
\texttt{Cost: 8 \\ Attack: 2 \\ Defense: 2 \\ Move: 2}
\\
Anti-submarine ship. Cancels most of submarine abilities.\\
With a destroyer in the fleet your airplanes can hit submarines.

\subsection*{Submarines}
\texttt{Cost: 6 \\ Attack: 2 \\ Defense: 1 \\ Move: 2}
\\
Can Surprise strike before every round of a seabattle.\\
Can Submerge (and retreat) before every round of a seabattle.\\
Two dices in convoy attacks.\\
Stealth movement (under enemy warships)\\
Can't hit air units. Can't be hit by air units.


\subsection*{Transports}
\texttt{Cost: 7 \\ Attack: 0 \\ Defense: 0 \\ Move: 2}
\\
Can transport 1 landunit plus 1 additional infantry\\
Can only chosen as a casualty if no other option is valid (\glqq chosen last\grqq).\\
Friendly units have to move on the trapo in their turn, be carried on your turn and offload in their future turns.

\noindent\begin{minipage}{\linewidth}
\section*{National Objectives}
\subsection*{Japan}
When Japan Isn't at War with US and/or UKAZ:
\begin{itemize}
\item 10 IPCs if Japan has not attacked French Indo-China, and has not made an unprovoked declaration of war against UKAZ. \textsl{Theme: Strategic resource trade with the US.}
\end{itemize}
When Japan Is at War with US and/or UKAZ:
\begin{itemize}
\item 5 IPCs if Japan controls the following territories: Guam, Midway, Wake Island, Gilbert Islands, and Solomon Islands. \textsl{Theme: Strategic outer defense perimeter.}
\item 5 IPCs per territory if Japan controls Calcutta, Sydney, Hawaii and/or San Francisco. \textsl{Theme: Major Allied power centers.}
\item 5 IPCs if Japan controls all of the following territories: Sumatra, Java, Borneo, and Celebes. \textsl{Theme: Strategic resource centers.}
\end{itemize}

\subsection*{ANZAC}
When ANZAC Is at War with Japan:
\begin{itemize}
\item 5 IPCs if an Allied power controls Malaya and ANZAC controls all of its original territories. \textsl{Theme: Malaya considered strategic cornerstone to Far East British Empire.}
\item 5 IPCs if the Allies (not including the Dutch) control Dutch New Guinea, New Guinea, New Britain, and the Solomon Islands. \textsl{Theme: Strategic outer defense perimeter.}
\end{itemize}
\end{minipage}
\columnbreak
\subsection*{China}
\begin{itemize}
\item 6 IPCs if the Burma Road is totally under allied control. China is also permitted to purchase artillery (represented by U.S. pieces). \textsl{Theme: Chinese military supply line corridor.}
\end{itemize}

\subsection*{United States}
When the United States Is at War:
\begin{itemize}
\item 30 IPCs if the US controls San Francisco. \textsl{Theme: Major shift from peacetime to wartime economy.}
\item 5 IPCs if the US controls both Alaska and Mexico.
\item 5 IPCs if the US controls all of the following territories: Aleutian Islands, Hawaiian Islands, Johnston Island, and Line Islands. \textsl{Theme: National sovereignty issues.}
\item 5 IPCs if the US controls Philippines. \textsl{Theme: Center of American influence in Asia.}
\end{itemize}

\subsection*{United Kingdom}
When the United Kingdom Is at War with Japan:
\begin{itemize}
\item 5 IPCs if the United Kingdom controls both Kwangtung
and Malaya. \textsl{Theme: Maintenance of the empire considered vital national objective.}
\end{itemize}

\end{multicols*}

\end{document}