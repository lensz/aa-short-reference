\documentclass[10pt,twoside]{article}
%
% Spaltenbreite soll 8cm sein (passt genau in AA Schachteln)
% A4 = 297x210, benutzen landscape
% 29,7/8 = 3,7125 -> 3 Spalten à 8cm + 5,7cm Rand..
% Sperator: 0,5cm
% margin-hori: 3,1
% Spaltenbreite 8cm -> 7,5 text 0,5 padding
% => 3*7,5 + 2*0.5 + 2*3,1 = 29,7 :)
%

\usepackage[landscape, a4paper, 
			left=3.1cm, 
			right=3.1cm,
			top=0.8cm, 
			bottom=1.2cm
			]{geometry}
\usepackage[utf8]{inputenc}
\usepackage[german]{babel}
\usepackage[T1]{fontenc}
\usepackage{multicol}
\usepackage{graphicx}
\usepackage{rotating}

\usepackage{lastpage}
\usepackage{fancyhdr} 

\addtolength{\footskip}{-0.4cm}
\fancyhf{}
\lfoot{Version: \today}
\cfoot{Short Reference for Axis \& Allies 1940 Pacific 2nd Edition}
\rfoot{\thepage / \pageref{LastPage}}
\pagestyle{fancy}    

\setlength{\columnsep}{0.5cm}
\setlength{\columnseprule}{0.4pt}

% damit beim SCI-Drucker beide Seiten übereinander sind (benötigt twosided layout)
\addtolength\oddsidemargin {0.25cm}

\begin{document}
\begin{multicols*}{3}

\section*{Turn Sequence}
\begin{enumerate}
\item Purchase and Repair

Ship repairs are free; Facility repairs cost 1~IPC per damage point 
\item Combat Move
\begin{enumerate}
\item Declare war?
\item Announce combat movements
\begin{enumerate}
\item Amphibious assaults
\item Strategic and tactical bombings
\item Normal combat moves
\end{enumerate}
\item Defender scrambling?
\end{enumerate}
\item Conduct Combat
\begin{enumerate}
\item Defender Kamikaze?
\item Strategic and tactical bombings
\item Amphibious assaults
\item General combat
\end{enumerate}
\item Noncombat Move
\begin{enumerate}
\item Landing of aircraft used in combat

allowed even on carriers being build this turn phase~5

\item Moving of troops that didn't engage in combat this turn
\end{enumerate}

\item Mobilize New Units

Place the units you bought in phase~1 on the map.

If you bought more units than you can place (eg. because of facility damage), you have to give units back. Price is refunded. Before placement check industry damage!
\item Collect Income
\begin{enumerate}
\item Convoy disruptions on you?
\item Check National Objectives
\item Calculate IPC for next turn
\end{enumerate}
\end{enumerate}
\columnbreak

% title 
\begin{center}
{\Large Short Reference}\\
\vspace{.2cm}
{\large Axis \& Allies 1940 Pacific 2nd Edition}

\vspace{.5cm}
contribute: https://github.com/lensz/aa-short-reference

\vspace{.5cm}
Version: \today
\end{center}

\section*{{\normalsize Sources}}
\begin{itemize}
\item Rulebook 2013 version
\item FAQ November 24, 2014
\end{itemize}

\noindent\textsl{Note that columns fit in the actual AA boxes..}

\section*{{\normalsize Abbreviations}}
\textbf{JP} Japan ~~ \textbf{US} USA ~~ \textbf{UK} United Kingdom ~~ \textbf{CH} China ~~ \textbf{AZ} ANZAC ~~ \textbf{UKAZ} UK and AZ

\begin{center}
\rule{6.5cm}{0.4pt}
\end{center}


\subsection*{General Combat}
\begin{enumerate}
\setcounter{enumi}{-1}
\item AAA's fire (only round 1) up to three times
\item Submarine surprise strike or submerge
\item Attacker fires
\item Defender fires
\item Push attack (goto 1) or retreat?
\end{enumerate}

\subsection*{Political Situation}
\begin{description}
\item[Axis Victory] JP controlling 6 victory cities (including Tokyo) for a complete round of play (So in beginning of JP turn). Liberation of Tokyo in this turn is also allowed.\textsuperscript{p.24}
\item[Allied Victory] Allies control one of their capitals plus Tokyo for complete a round. Liberation of a capital is also allowed. \textsuperscript{p.24}
\end{description}

%\subsection*{Politics}
At the beginning, there is only war between JP-CH. Allied troops in/through CH are considered an act of war against JP. CH units can only move on CH emblems plus Kwangtung and Burma -- even the fighter. No power on either side may enter sovjet territory.

If JP is at peace with US, JP may not end move within 2 territories of Alaska and West US. On the other hand, US may not end move in sea zone adjacent to JP-controlled territories. 

An unprovoked attack from JP to \{AZ, UK, Dutch\} results in war with whole Allies. And US gets 30 IPC immediately. If this does not happen, US can declare war on JP if both are still at peace at the beginning of the \glqq Collect Income\grqq\ phase in the third US turn. So in the end, US has every turn they are at war 30~IPC extra, independent from who declared war against whom.

JP attack on France has no consequences here. UKAZ is treated as one nation for political purposes. UKAZ may declare war on JP any time (but then without US).
UKAZ may take dutch territories for free right from the beginning.

%\section*{Miscellaneous}

\subsection*{Movement misc.}
\begin{itemize}
\item submarines/transports don't block movement like surface warships. But you have to ignore them all at once or attack them all and stop moving
\item submarines can move in/through hostile seazones if there is no destroyer present
\item fly over neutrals is not allowed until captured (even friendly neutral)
\end{itemize}

\subsection*{Blitzing}
Move two territories through hostile lands and capture the first one on the fly. You can only blitz through empty territory. Units -- even AAA, airports and industry -- stop the movement.\textsuperscript{p.14}

\subsection*{Surprise Strike and Submerge of submarines}
Attacking and defending submarines may choose one of the above before every round of the sea battle -- but only if the opposing side has no destroyer! Attacking player decides first what to do. Submarines that did a surprise strike don't fire in the normal combat again.

\subsubsection*{Submerge}
Submarines retreat immediately from battle. They return back to the map.

\subsubsection*{Surprise Strike}
Attacking surprise strike hits on 2 or less. Defending surprise strike hits on 1 or less. Can only hit sea units no air units. After the defending surprise strike, destroyed units get removed -- they don't shoot in the residual combat. Trapos can't be hitten, unless they are the only target. Can be done at every combat round as long as there is no destroyer opposing.

\subsection*{Scramble}
Is a move of the defender at the end of the \glqq Combat Move\grqq\ of the attacker. Attacker may not change something after scrambling is announced.

Scrambled planes may defend against enemy ships even if there are no friendly ships present. Friendly powers can scramble, too. With respect to the normal three aircraft per airbase rule.\textsuperscript{p.15}

\subsection*{Kamikaze}
Special defense strike against surface warships. Cannot hit transports and submarines. JP has to announce them before conducting combat. JP must declare which ship is attacked with how many strikes. Every token hits on 2 or less. Successful or not, the strike prevents offshore bombardment.\textsuperscript{p.16}

\subsection*{Amphibious Assaults}
All amphibious assaults have to be announcend before any combat. Attacker decides first which of his planes fight on sea and which on land. Later the defender decides as usual if he scrambles or not. Transports can only unload from a friendly seazone (no enemy surface warships). So mostly, the seazone has to be cleared in a seabattle before landing. While enemy submarines and transports may be ignored (they are not surface warships), your transports can only unload in sea zones with ignored submarines if there is at least one friendly warship present.
\begin{enumerate}
\item Sea Combat

With defending warships and/or scrambled planes. Defending submarines/transports may be ignored. But if there is combat, all units are involved.
\item Shorebombardement

Only if there was NO sea combat in the offloading seazone.  One ship per unloaded land unit may fire. Normal dice rolls for battleship and cruiser. Defeated land units may fire back in the first land combat round!
\item Land combat

Normal land combat, but landing units may not retreat.
\end{enumerate}

\subsection*{Convoy Disruptions}
At the beginning of your \glqq Collect Income\grqq\ phase, do enemy warships (all except carrier) and carrier based air units fire on your convoys. Only in convoy-provinces. Your own warships in the zone have no effect.\\
Submarines and air units have two dices, all other warships only one. Hits 4 or more are ignored. The total damage is the sum of all rolls with 3 or less. But you can only disrupt so much IPC as the neighbouring territories have!

\subsection*{Strategical and Tactical Bombing Raids}
Direct attack with tactical bombers on air- or navalbases. Strategic bombers can also attack industry.
Damaged facilities have to be repaired (with IPC). Defender and Attacker can both add fighters (but they don't have to). Defending fighters -- or interceptors -- come from the attacked territory. Attacking fighters -- or escort fighters -- from everywhere. Only if the defender uses interceptors, there follows a special airbattle with
\begin{itemize}
\item only one round of combat
\item all units have attack/defense value of 1
\end{itemize}
The fighters don't participate in the actual bombing, so no facility-AA fire on them. If a territory has multiple target facilities, the attacker may split his bombers. Each facility has its own AA.

Each complex rolls one dice against each bomber attacking it. The facility-AA hits on 1 or less and the bomber is removed immediately. Now the surviving bombers roll one dice each. Strategic bombers add 2 to each dice. The sum of those dice rolls is the total damage inflicted. Place one gray chip under the facility for every damage point to indicate damage. Max damage is 20 for major industry and 6 for everything else.\\
None of the involved planes can participate in other battles that turn.\textsuperscript{p.16}

\subsection*{Capturing/Liberating Capitals}
Liberate captured territory from your alliance gives it to the original controller if his capital is not in enemy hands. If capital is captured, you take it for yourself until the capital is free.

\textbf{Capture Capital:} Like normal territory. Plus you get all IPC the country has on its bank account. Original controller can't collect income and can't buy new units. He does only \glqq Combat\grqq\ and \glqq Non-Combat\grqq\ until his capital is liberated.

\textbf{Liberate Capital:} territories in hands of friends go immediately back to own control.

\pagebreak

\section*{Units}
\subsection*{Industry}
\texttt{Cost: minor 12, major 30, upgrade 20}\\
For building units.\\
On capture major one gets downgraded to minor.
Minor needs territory with value two or higherand may be build on captured territory. Major require 3 or higher and an original controlled territory. Both cannot be placed on islands. You cannot build units in friendly industry (even liberated).\\
Damaged industry produces less units (one for every damage point). Maximal damage is 6 respectively 20.

\subsection*{Air- and Navalbase}
\texttt{Cost: 15}\\
Allowes one additional movement point, repair of large ships (sea) and scrambling (air).\\
Carrier aircraft does not profit from airbase on land. Carriers and battleships get repaired in friendly navalbases in the turn of the shipowner.
\\You can use all bases in your alliance.\\
Becomes inoperative with three or more damage points. So range is reduced, no repairs are done and scrambling is prohibited.

\subsection*{Infantry}
\texttt{Cost: 3 \quad\quad\quad Attack: ~1 (2 with Art) \\ Move: 1 \quad\quad\quad Defense: 2 \\ }
Attack increases if paired with artillery.

\subsection*{Artillery}
\texttt{Cost: 4 \quad\quad\quad Attack: ~2 \\ Move: 1 \quad\quad\quad Defense: 2 \\ }
Can be paired with infantry and mechanized infantry.

\subsection*{Mechanized Infantry}
\texttt{Cost: 4 \quad\quad\quad Attack: ~1 (2 with Art) \\ Move: 2 \quad\quad\quad Defense: 2 \\ }
Attack increases if paired with artillery.\\
Can blitz when paired with a tank. Moving must start and end in same province for tank and mech.

\subsection*{Tanks}
\texttt{Cost: 6 \quad\quad\quad Attack: ~3 \\ Move: 2 \quad\quad\quad Defense: 3 \\ }
Can be paired with tactical bombers and mechanized infantry.\\
Can blitz. Allowes mechanized infantry to blitz.

\subsection*{AAA (Anti Aircraft Artillery)}
\texttt{Cost: 5 \quad\quad\quad\quad\quad\quad\quad ~Attack: ~- \\ Move: 1 (non-combat) \quad Defense: 0 \\}
Can not be used during attacks (except amphibious assault). Can not defend, but can be used as casualty.\\
Can defend against planes attacking the territory it stands in. They fire only once before the first actual combat round. Hits on 1. Each AAA may fire on 3 attacking airplanes, but on each airplane may only be shot once (so 5 fighters, 2 AAA $\rightarrow$ 5 rolls). Destroyed planes don't shoot back. Can \emph{not} defend against bombing runs -- facilities have there own AA-guns.

\subsection*{Fighters}
\texttt{Cost: 10 \quad\quad\quad Attack: ~3 \\ Move: ~4 \quad\quad\quad Defense: 4 \\ }
Can be used on a carrier.\\
Can be paired with a tactical bomber.\\
Can escort and intercept bombing raids (see \glqq Strategical and Tactical Bombing Raids\grqq)

\subsection*{Tactical Bombers}
\texttt{Cost: 11 \quad Attack: ~3 {\scriptsize(4 with tank or fighter)} \\ Move: ~4 \quad Defense: 3 \\ }
Can be used on a carrier.\\
Attack increases to 4 if paired with a fighter or a tank.\\
Can conduct bombing raids (see \glqq Strategical and Tactical Bombing Raids\grqq).

\subsection*{Strategic Bombers}
\texttt{Cost: 12 \quad\quad\quad Attack: ~4 \\ Move: ~6 \quad\quad\quad Defense: 1 \\ }
Can not be used on a carrier.\\
Can conduct bombing raids (see \glqq Strategical and Tactical Bombing Raids\grqq).


\subsection*{Battleships}
\texttt{Cost: 20 \quad\quad\quad Attack: ~4 \\ Move: ~2 \quad\quad\quad Defense: 4 \\ }
Has two lifes.\\
Can conduct offshore bombardment.

\subsection*{Carrier}
\texttt{Cost: 16 \quad\quad\quad Attack: ~0 \\ Move: ~2 \quad\quad\quad Defense: 2 \\ }
Has two lifes.\\
Guest planes are treated as cargo and move with the carrier but don't defend it. Own planes move independently from the carrier. So they launch instantly when the carrier moves. Planes can only land on the ship in the seazone the carrier ends its turn -- contrary to transports, which can load while moving.\\
Damaged carriers can't conduct air operations. Planes can't land on it and friendly (cargo) planes are trapped on it and can't move until repaired.

\subsection*{Cruiser}
\texttt{Cost: 12 \quad\quad\quad Attack: ~3 \\ Move: ~2 \quad\quad\quad Defense: 3 \\ }
Can conduct offshore bombardment.

\subsection*{Destroyer}
\texttt{Cost: 8 \quad\quad\quad Attack: ~2 \\ Move: 2 \quad\quad\quad Defense: 2 \\ }
Anti-submarine ship. Cancels most of submarine abilities.\\
With a destroyer in the fleet your airplanes can hit submarines.

\subsection*{Submarines}
\texttt{Cost: 6 \quad\quad\quad Attack: ~2 \\ Move: 2 \quad\quad\quad Defense: 1 \\ }
Can surprise strike and submerge (direct retreat) before every round of a seabattle.\\
Two dices in convoy attacks.\\
Stealth movement (under enemy warships).\\
Can't hit air units. Can't be hitten by air units, unless there is a destroyer present.

\subsection*{Transports}
\texttt{Cost: 7 \quad\quad\quad Attack: ~0 \\ Move: 2 \quad\quad\quad Defense: 0 \\ }
Can transport 1 land unit plus 1 additional infantry unit.\\
Can only be chosen as a casualty if no other option is valid (\glqq chosen last\grqq).\\
Friendly units have to move on the transpoter in their turn, be carried on your turn and offloaded in their future turns.

\begin{center}
\begin{sideways}
\scalebox{0.76}{
\bgroup\def\arraystretch{1.1}\begin{tabular}{l|c|c|c|c|l}
Name & Cost & Att & Def & Move & Special \\
\hline
Infantry & 3 & 1 (2) & 2 & 1 & -- \\
Artillery & 4 & 2 & 2 & 1 & -- \\
Mech. Inf. & 4 & 1 (2) & 2 & 2 & can blitz with tank \\
Tank & 6 & 3 & 3 & 2 & can blitz \\
AAA & 5 & -- & 0 & 1 & shoots on up to 3 planes \\
\hline
Fighter & 10 & 3 & 4 & 4 & escort/intercept bombings \\
Tac. Bomber & 11 & 3 (4) & 3 & 4 & Tactical bombings \\
Str. Bomber & 12 & 4 & 1 & 6 & Strategic bombings \\
\hline
Transport & 7 & 0 & 0 & 2 & transports 1 unit + 1 inf. \\
Submarine & 6 & 2 & 1 & 2 & surprise strike \& submerge \\
Destroyer & 8 & 2 & 2 & 2 & Anti-submarine ship \\
Cruiser & 12 & 3 & 3 & 2 & Shore bombardment \\
Carrier & 16 & 0 & 2 & 2 & 2 lifes \\
Battleship & 20 & 4 & 4 & 2 & 2 lifes; shore bombardment\\
\hline
Min. Industry & 12 & \multicolumn{4}{l}{build up to 3 units, but 1 unit less per dmg point} \\
Maj. Industry & 30 & \multicolumn{4}{l}{build up to 10 units, but 1 unit less per dmg point} \\
Ind. Upgrade & 20 & \multicolumn{4}{l}{upgrade minor to major industry} \\
Airbase & 15 & \multicolumn{4}{l}{+1 movement, scramble, inoperative at 3+ dmg} \\
Navalbase & 15 & \multicolumn{4}{l}{+1 movement, rapair, inoperative at 3+ dmg} \\
\end{tabular}
\egroup
}
\end{sideways}
\end{center}

\noindent\begin{minipage}{\linewidth}
\section*{National Objectives}
\subsection*{Japan}
When Japan is not at War with US:
\begin{itemize}
\item 10 IPCs if Japan has not attacked French Indo-China, and has not made an unprovoked declaration of war against UKAZ. 

\textsl{Theme: Strategic resource trade with the United States.}
\end{itemize}
When Japan is at War with US and/or UKAZ:
\begin{itemize}
\item 5 IPCs if Japan controls the following territories: Guam, Midway, Wake Island, Gilbert Islands, and Solomon Islands.

\textsl{Theme: Strategic outer defense perimeter.}
\item 5 IPCs per territory if Japan controls Calcutta, Sydney, Hawaii and/or San Francisco.

\textsl{Theme: Major Allied power centers.}
\item 5 IPCs if Japan controls all of the following territories: Sumatra, Java, Borneo, and Celebes.

\textsl{Theme: Strategic resource centers.}
\end{itemize}

\subsection*{ANZAC}
When ANZAC Is at War with Japan:
\begin{itemize}
\item 5 IPCs if an Allied power controls Malaya and ANZAC controls all of its original territories.

\textsl{Theme: Malaya considered strategic cornerstone to Far East British Empire.}
\item 5 IPCs if the Allies (not including the Dutch) control Dutch New Guinea, New Guinea, New Britain, and the Solomon Islands.

\textsl{Theme: Strategic outer defense perimeter.}
\end{itemize}
\end{minipage}
\columnbreak

\subsection*{United Kingdom}
When the United Kingdom Is at War with Japan:
\begin{itemize}
\item 5 IPCs if the United Kingdom controls both Kwangtung
and Malaya.

\textsl{Theme: Maintenance of the empire considered vital national objective.}
\end{itemize}

\subsection*{United States}
When the United States Is at War:
\begin{itemize}
\item 30 IPCs if the US controls San Francisco.

\textsl{Theme: Major shift from peacetime to wartime economy.}
\item 5 IPCs if the US controls both Alaska and Mexico.
\item 5 IPCs if the US controls all of the following territories: Aleutian Islands, Hawaiian Islands, Johnston Island, and Line Islands.

\textsl{Theme: National sovereignty issues.}
\item 5 IPCs if the US controls Philippines. 

\textsl{Theme: Center of American influence in Asia.}
\end{itemize}

\subsection*{China}
\begin{itemize}
\item 6 IPCs if the Burma Road is totally under allied control. China is also permitted to purchase artillery (represented by U.S. pieces).

\textsl{Theme: Chinese military supply line corridor.}
\end{itemize}

\end{multicols*}

\end{document}